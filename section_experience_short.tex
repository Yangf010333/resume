% Awesome CV LaTeX Template
%
% This template has been downloaded from:
% https://github.com/huajh/huajh-awesome-latex-cv
%
% Author:
% Junhao Hua


%Section: Work Experience at the top
\sectionTitle{项目经历}{\faCode}
 
\begin{experiences}
  
  \experience
  {2018年9月}   {物体检测的数据集增广}{Python编程}{}
  {2019年5月} {
  	\begin{itemize}
  		\item 通过交换相同类别的物体来合成图片;
  		\item 通过选择性重采样解决MSCOCO数据集的类别不均衡现象;
  		\item 为物体检测模型加入类别均衡损失提升性能;
  		\item 通过风格转换方法提升合成图片的真实度;
  		\item 通过图像复原修复合成图片的"空洞"现象;
  		\item \link{https://github.com/Yangf010333/Augmentation}{代码链接};
  		\item \link{https://arxiv.org/abs/1906.00358}{论文链接};
  	\end{itemize}
  }
  {物体检测,数据增广}
  \emptySeparator
  
  \experience
  {2018年9月}   {哈工大校园社交网络系统}{Python,css,html,js编程}{}
  {2018年12月} {
      \begin{itemize}
      \item 实现用户的注册,登陆,对话,个人经历的建立与发布;
      \item 日志文件的创建,保存,修改,删除,分享,评论;
      \item 实现包括各种条件的检索:时间,关键字,作者和全文检索;
      \item 项目部署到服务器,并维护;
      \item 项目扩展为chrome浏览器插件;
                                                                                       
      \end{itemize}
  }
  {Tomcat,全文检索,css,js,chrome浏览器插件开发}
  \emptySeparator

 

  \experience
  {2018年2月}   {美国大学生数学建模竞赛E题}{Python编程}{}
  {2018年2月} {
  	\begin{itemize}
  		\item 找到明确的计算方法,刻画气候和国家脆弱程度之间的相似性;
  		\item 编程演算气候对脆弱国家和非脆弱国家的影响;
                                                                                     
  	\end{itemize}
  }
  {Python,相似性度量,回归模型}
  \emptySeparator
	
   \experience
  {2017年9月}   {ROS基本功能在FreeRTOS上的实现}{C/C++编程}{}
  {2017年12月} {
  	\begin{itemize}
  		\item 在FreeRTOS实时操作系统上实现非实时操作系统ROS的基本功能;
  		\item 移植ROS的Publisher,Listener,Messages,Topics到FreeRTOS上;
  		\item 移植LWIP到FreeRTOS上;
  		                                                                               
  	\end{itemize}
  }
  {操作系统, ROS, STM32, LWIP}
  \emptySeparator

  \experience
  {2017年9月}   {全国大学生数学建模竞赛B题}{matlab/C++编程}{}
  {2017年9月} {
    \begin{itemize}
      \item 实现K-means聚类算法对数据实现聚类;
      \item 实现SVM分类算法找到最优分类面;
                                                                                      
      \end{itemize}
  }
  {K-means, SVM}
  \emptySeparator


\end{experiences}
